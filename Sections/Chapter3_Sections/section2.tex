\section{训练模型的简易流程}

% ----------------------------------------------------------------------------------------- %
\subsection{基本流程}
通常深度学习领域,我们将


\begin{description}
	\item[1.] 数据准备
	\begin{description}
		\item[1.1] 数据收集
		\begin{itemize}
			\item 下载、抓取或者合成数据
			\item 公开数据集(ImageNet、COCO、MNIST 等)
			\item 定义数据集(遥感图像、医学图像、草图数据等)
			\item 数据增强(旋转、翻转、模糊等)
		\end{itemize}
		\item[1.2] 数据预处理
		\begin{itemize}
			\item 归一化
			\item 格式转换
			\item 数据储存形式等
		\end{itemize}
		\item[1.3] 数据加载
		\begin{itemize}
			\item 定义批处理
		\end{itemize}
	\end{description}
	\item[2.] 定义模型
	\begin{itemize}
		\item 选择合适的架构
		\item 决定模型的层数,参数量
		\item 加载预训练模型
	\end{itemize}
	\item[3.] 定义损失函数和优化器
	\begin{description}
		\item[3.1] 选择损失函数 
		\begin{itemize}
			\item 分类任务:交叉熵,折页损失函数(SVM)
			\item 回归任务:均方差(MSE)
			\item 对象检测或者分割:交并比(ioU)
			\item 策略优化:KL 散度
			\item 词嵌入:噪音对比估计(NCE)
			\item 词向量:余弦相似度
			\item 等等
		\end{itemize}
		\item[3.2] 选择优化器
		\begin{itemize}
			\item SGD
			\item AdamW
		\end{itemize}
		\item[3.3] 学习率的调度
		\begin{itemize}
			\item StepLR
			\item Cosine Annealing
		\end{itemize}
	\end{description}
	\item[4.] 训练模型
	在每个 epoch 中
	\begin{description}
		\item 训练阶段
		\begin{itemize}
			\item 前向传播
			\item 计算损失
			\item 反向传播
		\end{itemize}
		\item 验证阶段
		\begin{itemize}
			\item 计算在验证集上的损失
			\item 评估指标
		\end{itemize}
	\end{description}
	\item[5.] 评估和测试
	\begin{itemize}
		\item 计算指标
		\item 绘制loss曲线
		\item 生成可视化结果
	\end{itemize}
	\item[6.] 模型保存和记录
	\begin{description}
		\item[6.1] 保存最佳模型
		\item[6.2] 记录训练信息
		\begin{itemize}
			\item 日志信息
			\item wandb
			\item 记录训练时长
			\item 记录模型的显存峰值
			\item 等等
		\end{itemize}
	\end{description}
	\item[7.] 部署和推理
	\begin{itemize}
		\item 导出 ONNX/TensorRT python
		\item 加载模型进行推理
	\end{itemize}
	\item[8.] 超参数调优
\end{description}

% ----------------------------------------------------------------------------------------- %
\subsection{基本代码模块详解}

% 1.
\subsubsection{A.数据准备}
在数据处理、编程开发或机器学习等领域,数据加载(Data Loading) 是指将数据从存储位置(如文件、数据库、网络接口等)读取到程序的内存中,使其成为可被程序直接访问和处理的格式的过程。它是数据处理流程中的基础环节,为后续的清洗、分析、建模等操作提供数据支持。通常采用:
\begin{itemize}
	\item RGB数据、OR BGR数据
	\item JPEG编码后的数据
	\item torchvision.datasets中shuju 
	\item torch.utils.data下的Dataset, Dataloader自定义的数据集
\end{itemize}

在机器学习(尤其是计算机视觉、自然语言处理等领域)中,数据增强(Data Augmentation) 是一种通过对原始训练数据进行合理的、有策略的变换或扩展,生成新的 “虚拟样本”,从而扩大训练数据集规模、丰富数据多样性的技术。
其核心目标是:让模型在训练时接触到更多样化的输入,增强模型对数据中各种变化(如噪声、变形、视角差异等)的鲁棒性,减少过拟合(即模型过度依赖训练数据的细节,在新数据上表现不佳)。
通常采用:

\begin{lstlisting}[language=python,caption={数据增强格式},label=code:Data Augmentation]
	import 
	
	# 将需要的变换均置于Compose盒子里面逐个处理
	train_transforms = transforms.Compose([
		transforms.RandonResizeCrop((227,227))
		transforms.ToTensor(),
		# 等等
	])
\end{lstlisting}
% 2.
\subsubsection{B.定义模型}

% 3.
\subsubsection{C.定义损失函数和优化器}

% 4.
\subsubsection{D.训练模型}

% 5. 
\subsubsection{E.训练和测试}





% ----------------------------------------------------------------------------------------- %
\subsection{定制代码模块详解}


\newpage